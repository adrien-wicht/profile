\documentclass[letterpaper,10.8pt]{article}

\usepackage{latexsym}
\usepackage[empty]{fullpage}
\usepackage{titlesec}
\usepackage{marvosym}
\usepackage[usenames,dvipsnames]{color}
\usepackage{verbatim}
\usepackage{enumitem}
\usepackage[pdftex]{hyperref}
\usepackage{fancyhdr}
\usepackage{booktabs}


\pagestyle{fancy}
\fancyhf{} % clear all header and footer fields
\fancyfoot{}
\renewcommand{\headrulewidth}{0pt}
\renewcommand{\footrulewidth}{0pt}

% Adjust margins
\addtolength{\oddsidemargin}{-0.375in}
\addtolength{\evensidemargin}{-0.375in}
\addtolength{\textwidth}{1in}
\addtolength{\topmargin}{-.5in}
\addtolength{\textheight}{1in}

\urlstyle{rm}

\raggedbottom
\raggedright
\setlength{\tabcolsep}{0in}

% Sections formatting
\titleformat{\section}{
  \vspace{-3pt}\scshape\raggedright\large
}{}{0em}{}[\color{black}\titlerule \vspace{-5pt}]

%-------------------------
% Custom commands
\newcommand{\resumeItem}[2]{
  \item\small{
    \textbf{#1}{: #2 \vspace{-2pt}}
  }
}

\newcommand{\resumeItemWithoutTitle}[1]{
  \item\small{
    {\vspace{-2pt}}
  }
}

\newcommand{\resumeSubheading}[4]{
  \vspace{-1pt}\item
    \begin{tabular*}{0.97\textwidth}{l@{\extracolsep{\fill}}r}
      \textbf{#1} & #2 \\
      \textit{\small#3} & \textit{\small #4} \\
    \end{tabular*}\vspace{-5pt}
}


\newcommand{\resumeSubItem}[2]{\resumeItem{#1}{#2}\vspace{-4pt}}

\renewcommand{\labelitemii}{$\circ$}

\newcommand{\resumeSubHeadingListStart}{\begin{itemize}[leftmargin=*]}
\newcommand{\resumeSubHeadingListEnd}{\end{itemize}}
\newcommand{\resumeItemListStart}{\begin{itemize}}
\newcommand{\resumeItemListEnd}{\end{itemize}\vspace{-5pt}}

%-------------------------------------------
%%%%%%  CV STARTS HERE  %%%%%%%%%%%%%%%%%%%%%%%%%%%%


\begin{document}

%----------HEADING-----------------
\begin{tabular*}{\textwidth}{l@{\extracolsep{\fill}}r}
  \textbf{{\LARGE Adrien Wicht}} & Email : \href{mailto:adrien.wicht@eui.eu}{adrien.wicht@eui.eu}\\
  & Mobile : +41 (0)79 532 22 08 \\
  \href{https://adrien-wicht.github.io/profile/}{Personal Website} \\
\end{tabular*}

%
%--------RESEARCH INTERESTS------------
\section{Research Interests}
	{ International Macroeconomics, Sovereign Debt and Defaults, Optimal Contracts.}

%
%--------REFERENCES------------
\section{References}
	
\begin{tabular}{lr}
% Referee 1
\begin{minipage}[t]{3in}
Ramon Marimon \\
European University Institute \\
 \href{mailto:ramon.marimon@eui.eu}{ramon.marimon@eui.eu} \\
+39 0554685911
\end{minipage}
&
% Referee 2
\begin{minipage}[t]{3in}
Alexander Monge-Naranjo \\
European University Institute \\
 \href{mailto:alexander.monge-naranjo@eui.eu}{alexander.monge-naranjo@eui.eu} \\
+39 0554685942
\end{minipage}

% Additional newline for spacing.
% Referee 3
\begin{minipage}[t]{3in}
Alain Gabler \\
Swiss National Bank\\
\href{mailto:alain.gabler@snb.ch}{alain.gabler@snb.ch}\\
+41 586313498\\
\end{minipage}
\end{tabular}

%-----------EDUCATION-----------------
\section{Education}
  \resumeSubHeadingListStart
    \resumeSubheading
      {European University Institute}{Florence, IT}
      {PhD in Economics}{08.19-- Current}
      
	   {\scriptsize \textit{Advisors: Ramon Marimon (1st) and Alexander Monge-Naranjo (2nd).}}
	   
      \resumeSubheading
      {University of Pennsylvania}{Philadelphia, US}
      {Visiting Scholar}{09.21 -- 02.22}
      
	   {\scriptsize \textit{Sponsor: Dirk Krueger.}}
	    
     \resumeSubheading
      {European University Institute}{Florence, IT}
      {MRes in Economics}{08.18 -- 07.19}
      
      \resumeSubheading
      {University of Zurich}{Zurich, CH}
      {MA in Economics}{09.15 -- 07.17}
      
       \resumeSubheading
      {University of Fribourg}{Fribourg, CH}
      {BA in Economics}{09.11 -- 06.14}
      
       \resumeSubheading
      {Coll\`{e}ge Sainte-Croix}{Fribourg, CH}
      {Maturity with Major in Law and Economics}{08.07 -- 07.11}

  \resumeSubHeadingListEnd
  
%
%--------ADDITIONAL COURSES------------
\section{Additional Coursework}
      
\resumeSubHeadingListStart

       \resumeSubheading
      {Florence School of Banking and Finance}{Florence, IT}
      {Sovereign Debt Risks}{23.05.22 -- 25.05.22}
      
      {\scriptsize \textit{Instructor: Lee Buchheit, Mitu Gulati and Ugo Panizza.}}
      
      \resumeSubheading
      {Study Center Gerzensee}{Gerzensee, CH}
      {An Introduction to Macro-Finance}{02.05.22 -- 06.05.22}
      
      {\scriptsize \textit{Instructor: Atif Mian.}}
      
      
	\resumeSubheading
      {University of Pennsylvania}{Philadelphia, US}
      {International Macroeconomics with Financial Frictions}{08.21 -- 12.21}
      
      {\scriptsize \textit{Instructor: Enrique Mendoza.}}
      
       \resumeSubheading
      {University of Pennsylvania}{Philadelphia, US}
      {Topics in Macroeconomic Theory}{08.21 -- 12.21}
      
      {\scriptsize \textit{Instructor: Alessandro Dovis.}}

\resumeSubHeadingListEnd


%-----------RA-----------------
\section{Research Assistance}
  \resumeSubHeadingListStart
    \resumeSubheading
    {Robert Schuman Center of Advance Studies, European University Institute}{Florence, IT}
    {Advisor: Ramon Marimon}{09.20 -- 02.21}
    \begin{itemize}
        \item{Development and computation of dynamic macroeconomic models.}
    	 \item{Data gathering, software programming and calibration.}
     \end{itemize}
      
	\resumeSubheading
    {International Policy Analysis Unit, Swiss National Bank}{Zurich, CH}
    {Head of unit: Alain Gabler }{07.17 -- 07.18}
    \begin{itemize}
        \item{Data processing and analysis.}
    	 \item{Database management.}
	 \item{Involvement in policy-relevant research.}
     \end{itemize}
\resumeSubHeadingListEnd


%-----------TA-----------------

\section{Teaching Assistance}
  \resumeSubHeadingListStart
   %\resumeSubheading
   % {European University Institute}{Florence, IT}
    %{Instructor: Alexander Monge-Naranjo}{04.23 -- 05.23}
   % \resumeItemListStart
   %     \resumeItem{Macroeconomics III -- Part. 2: Incomplete Markets}{PhD core course}
    %  \resumeItemListEnd
        
         \resumeSubheading
    {European University Institute}{Florence, IT}
    {Instructor: Ramon Marimon}{09.22 -- 10.22}
    \resumeItemListStart
        \resumeItem{Advanced Macro Introduction}{PhD elective course}
     \resumeItemListEnd
        
    \resumeSubheading
    {European University Institute}{Florence, IT}
    {Instructor: Jes\'{u}s Bueren}{01.20 -- 03.20}
    \resumeItemListStart
        \resumeItem{Macroeconomics II}{PhD core course}
      \resumeItemListEnd
      
	  \resumeSubheading
    {European University Institute}{Florence, IT}
    {Instructor: Ramon Marimon}{04.20 -- 06.20}
    \resumeItemListStart
        \resumeItem{Macro-Finance and Policy Design}{PhD elective course}
      \resumeItemListEnd
\resumeSubHeadingListEnd

%-----------WORKINPROGRESS-----------------
\section{Work in Progress}
\resumeSubHeadingListStart
\resumeSubItem{Efficient Sovereign Debt Management: The role of History, Maturity, Buyback and Default}{This paper identifies the role of past history, buybacks and defaults in the context of constrained efficient sovereign borrowing. I derive a market economy in which a sovereign borrower trades non-contingent bonds of different maturities with a foreign lender. The borrower is relatively impatient and lacks commitment. I show that the market economy cannot implement the Planner's constrained efficient allocation through defaults but instead by changes in maturity and debt buybacks. Especially, when the borrower is sufficiently patient Markov strategies can implement the Planner's allocation in steady state. Otherwise, history-dependent strategies are required. Nevertheless, interpreting the impatient borrower as a shot-run player, small perturbations in the payoff of the market participants rule out any other strategies than Markov ones. In this case, the constrained efficient allocation represents an \textit{ideal type} which can only be approximated by the market economy through history-invariant debt management policies. Argentina and Brazil present evidence of such approximation albeit with different policies and outcome.  }
\resumeSubItem{Making Sovereign Debt Safe with a Financial Stability Fund (joint with Yan Liu and Ramon Marimon)}{We develop an optimal design of a Financial Stability Fund that coexists with the international debt market. The sovereign can borrow long-term defaultable bonds on the private international market, while having with the Fund a long-term contingent contracts subject to limited enforcement constraints. There is a contract that minimizes the debt absorbed by the Fund, guaranteeing full debt stabilization. In equilibrium, the seniority of the Fund contract, with respect to the privately held debt, is irrelevant. We calibrate our model to the Italian economy and show it would have had a more efficient debt accumulation with the Fund.}
%\resumeSubItem{The Generalized Euler Equation and the Bankruptcy-Sovereign Default Problem (joint with Xavier Mateos-Planas, Sean McCrary and Jose-Victor Rios-Rull)}{We first revisit the standard case with short term debt and no commitment (Arellano (2008)) and describe the theoretical properties as characterized by (Clausen and Strub (2020)). They involve an optimal saving decision, which is a Generalized Euler Equation (GEE) (or Euler Equation with derivatives of future actions). No expression for price derivatives is needed. We then characterize the equilibrium with long term debt, showing the properties of the price functions and giving a formulation of the Markov optimality conditions in the long-term debt case that does not rely on price derivatives.}
\resumeSubItem{Seniority and Sovereign Default}{This paper documents the implications of the de facto seniority structure of sovereign debt. Empirically, it presents strong evidence that defaults involving multilateral creditors are infrequent, last relatively longer and are associated with greater private creditors' losses. It subsequently builds a model of endogenous defaults and renegotiations to rationalize those findings. There, the multilateral debt generates an important pecuniary externality. While it can work as a commitment device and dampen the default risk, it raises the subordination risk of private liabilities. Furthermore, a tough renegotiation strategy ensures the multilateral lenders to safeguard a lending policy at preferential rates. It also rationalizes the longer default's duration and the greater private creditors' losses observed in default episodes implicating multilateral lenders.}
\resumeSubItem{The Generalized Euler Equation and the Bankruptcy-Sovereign Default Problem (joint with Xavier Mateos-Planas, Sean McCrary and Jose-Victor Rios-Rull)}{(abstract and draft available soon)}
\resumeSubHeadingListEnd

%-----------PUBLISHED-----------------
\section{Published Work}
\resumeSubHeadingListStart
\resumeSubItem{Efficient Sovereign Debt Management: The role of History, Maturity, Buyback and Default (joint with Joschka Gerigk and Miriam Rinawi), Aussenwirtschaft, Vol. 69(1), pp. 45-76}{This paper investigates the relationship between demographics and the current account. We analyze the impact of recent demographic changes and provide a forecast of its future impact. Overall, we find a strong and robust, non-linear demographic effect. In particular, we find a positive association between the current account and the share of a population’s prime-age individuals and a negative association with the share of the elderly. Our forecast suggests that, given the dramatically aging population in most industrialized countries, demographics will likely decrease the current account balance in the near future in those countries.}
\resumeSubHeadingListEnd

%-----------POLICYPAPERS-----------------
\section{Policy Papers}
\resumeSubHeadingListStart
\resumeSubItem{Euro Area fiscal policies and capacity in post-pandemic times (joint with Ramon Marimon)}{The main legacy of the post-Covid-19-crisis euro area fiscal framework should be the development of a unique integrated fiscal policy and of a permanent and independent Fiscal Fund to implement it. To arrive at this conclusion, we analyse the challenges and build on current research on the optimal design of a fiscal fund. We characterise the fiscal policy, and the development of the Fund, together with the role and form that the Stability and Growth Pact can take in the new fiscal framework.}
\resumeSubHeadingListEnd

%-----------CONFERENCES-----------------
\section{Conferences, Seminars and Workshops}
\resumeSubHeadingListStart
\resumeSubItem{Society of Economic Dynamics annual Meeting}{University of Madison-Wisconsin, 28-30.05.22}
\resumeSubItem{ADEMU Workshop}{BSE Summer Forum, 15-16.05.22}
\resumeSubItem{5th Interdisciplinary Sovereign Debt Research and Management Conference (DebtCon5)}{EUI, Georgetown University, The Graduate Institute of Geneva, Sovereign Debt Forum, 25-26.05.22}
\resumeSubItem{21st Doctoral Meeting in International Trade, International Macroeconomics and Urban Economics}{Sciences Po, 28-29.04.22}
\resumeSubItem{Philadelphia Fed Seminar}{Federal Reserve Bank of Philadelphia, 16.02.22.}
\resumeSubItem{Macro Lunch Seminar}{University of Pennsylvania, 22.10.21.}
\resumeSubItem{A Dynamic Economic and Monetary Union (ADEMU) Workshop}{European Stability Mechanism, EUI and University of Luxembourg, 08-09.10.20}
\resumeSubHeadingListEnd

%-----------AWARDS-----------------
\section{Awards and Grants}
\begin{description}[font=$\bullet$]
\item {Doc Mobility Grant, Swiss National Science Foundation, 2021-2022.} 
\item {EUI PhD Grant, Swiss Grant-awarding Authority, 2018-2022.} 
\item {The Ernst \& Young Award, University of Fribourg, Best Bachelor of Arts in Economics, 2014.} 
\item {Prix de la Banque Cantonale de Fribourg, Coll\`{e}ge Sainte-Croix, Best GPA in Law and Economics section, 2011.}
\item {Prix du Fond Tesch, Coll\`{e}ge Sainte-Croix, Best GPA in the French-speaking section, 2011.}
\end{description}


%
%--------PROGRAMMING SKILLS------------
\section{Skills Summary}
	\resumeSubHeadingListStart
	\resumeSubItem{Languages}{French (native), German (fluent), Swiss German (fluent), English (fluent), Italian (intermediate).}
	\resumeSubItem{Text editors}{Latex, MS Office and Apple applications.}
	\resumeSubItem{Software}{Excel VBA, Fortran, Matlab, MySQL, OpenMP, QGIS, R, Stata.}
\resumeSubHeadingListEnd

%-------------------------------------------
\end{document}